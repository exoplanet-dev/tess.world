\documentclass[modern]{aastex62}

\input{xostyle}
\input{xosymbols}

\begin{document}\raggedbottom\sloppy\sloppypar\frenchspacing

\title{%
  {\bf NOTE: this is a very rough draft of this manuscript --- read at your own risk!}
  \\
  tess.world: An open source living catalog of TESS planet fits
}

\author[0000-0002-9328-5652]{Daniel Foreman-Mackey}
\affil{Center for Computational Astrophysics, Flatiron Institute, 162 5th Ave, New York, NY 10010}

\author{others}
\noaffiliation

\begin{abstract}
  \noindent
  This is sick.
\end{abstract}

\keywords{%
  methods: data analysis ---
  methods: statistical
}

\section{Introduction}



\citep{Foreman-Mackey:2017}

\section{Parameterization}

There are many options for parameterization of a transit, but we choose to model these transits as circular orbits parameterized by their observables.
We also track the relevant Jacobians.

The main parameters are:
\begin{enumerate}
  \item \emph{two reference transit times}, one near the beginning of the observations, $t_0$, and one near the end, $t_{-1}$, measured in \tess\ BJD,
  \item \emph{the approximate transit depth}, $\delta$, measured in parts-per-thousand,
  \item \emph{the impact parameter} of the orbit, $b$, constrained to be $|b| \le 1$, and
  \item \emph{the transit duration}, $\tau$, measured in hours.
\end{enumerate}

With more discussion of the details, motivations, and limitations below.

\paragraph{Transit times}
To speed up the analysis, we assume that the discovery period and phase of the orbit are close enough to the truth that we can fit only the data near the expected transit times.
One consequence of this assumption is that we are assuming that the \emph{number} of periods that occur in the \tess\ observational baseline is correct.
Practically, this means that our prior assumption is that the transits must occur within the data cutouts.
This can be difficult to enforce---especially for low signal-to-noise transits---but a good approximation can be achieved by fitting for two reference transit times, $t_0$ and $t_{-1}$, with a fixed number of periods, $N_P$, between them, instead of a single reference time and the period.
Then we can compute the implied period as $P = (t_{-1} - t_0) / N_P$.
Importantly this does not change the prior on $P$ and $t_0$ since the Jacobian is a constant $1/N_P$.

\paragraph{Transit depth}
It is worth spending a moment on the transit depth parameterization.
This choice of parameterization leads to efficient computation and convergence, but it comes with non-trival shortcomings.
Since the physical parameter that is required to compute the light curve model is the radius ratio between the planet and the star $k = R_\mathrm{p} / R_\star$, we need to choose a parameterization that is invertible and that isn't generally possible.
In some cases, using radius ratio directly as the parameter can work well, but \dfmtodo{explain cases where it's not}.
Instead, we choose to parameterize the approximate transit depth $\delta$ using the small planet approximation.
This is useful because it is directly invertible (conditioned on the limb darkening parameters and impact parameter), but it restricts us to considering non-grazing transits with impact parameter $|b| \le 1$.
Accepting this restriction, we can compute the approximate transit depth for a limb darkened light curve by assuming that the intensity of the star is uniform under the disk of the planet.
For quadratic limb darkening, the intensity profile is
\begin{equation}
  I(r) = 1 - u_1\,[1 - \mu(r)] - u_2\,[1 - \mu(r)]^2
\end{equation}
where $\mu(r) = \sqrt{1 - r^2}$.
The ratio of the occulted flux to the total stellar flux when the transit is deepest ($r = b$) is \citep[the same results are discussed by][]{Mandel:2002,Csizmadia:2013}
\begin{eqnarray}
  \delta &\approx& \frac{\int_0^k\,2\,\pi\,r\,I(b)\dd r}{\int_0^1\,2\,\pi\,r\,I(r)\dd r} \nonumber\\
  &=& \frac{k^2\,\left(1 - u_1\,[1 - \mu(b)] - u_2\,[1 - \mu(b)]^2\right)}{1 - u_1/3 - u_2/6}\quad.
\end{eqnarray}
Therefore, since $k$ must be positive, we have a one-to-one transformation between $\delta$ and $k$ conditioned on impact parameter $|b| \le 1$ and the limb darkening coefficients.
It is also important to include the Jacobian factor so that fitting in $\delta$ doesn't introduce a strange prior on $r$.
In this case, the relevant factor is
\begin{equation}
  \left|\frac{\dd k}{\dd \delta}\right| = \left|\frac{1 - u_1/3 - u_2/6}{2\,k\,\left(1 - u_1\,[1 - \mu(b)] - u_2\,[1 - \mu(b)]^2\right)}\right| \quad.
\end{equation}

\paragraph{Impact parameter}
Constrained to be non-grazing.
\dfmtodo{Discuss the consequences of this.}

\paragraph{Transit duration}
The physical parameter required for computing the transit model is the semi-major axis, $a$, in units of the stellar radius, but the transit duration $\tau$ is better constrained so it can be better as a fit parameter.
For a circular orbit, the transit duration is \citep{Winn:2010}
\begin{equation}
  \tau = \frac{P}{\pi}\,\sin^{-1}\left( \frac{\sqrt{(1 + k^2) - b^2}}{a\,\sin i} \right) \quad.
\end{equation}
Rearranging this, we find
\begin{equation}
  a^2\,\sin^2 i\,\sin^2\left(\frac{\pi\,\tau}{P}\right) = (1 + k^2) - b^2 \quad.
\end{equation}
Then, using the fact that $\cos^2 i = b^2 / a^2$, we find
\begin{equation}
  a^2 = \frac{(1 + k)^2 - b^2\,\cos^2\phi}{\sin^2\phi}
\end{equation}
for $\phi = \pi\,\tau / P$.
And the Jacobian is
\begin{eqnarray}
  \frac{\dd a}{\dd \tau} &=& \frac{\pi\,\cos \phi}{a\,P\,\sin^3 \phi}\,\left[b^2 - (1 + k)^2\right] \quad.
\end{eqnarray}

Finally, from the period and semi-major axis, we can compute the implied stellar density (under this assumption of a circular orbit)
\begin{equation}
  \rho_\mathrm{circ} = \frac{3\,\pi\,a^3}{G\,P^2} \quad.
\end{equation}
It is important to note that this is not necessarily the same as the actual stellar density and that, in a multi-planet system, this implied density won't be the same for each planet \citep[see, for example,][]{Dawson:2012, Kipping:2012}.

\section{Single transits}



\acknowledgements

This project was developed in part at the Building Early Science with TESS meeting, which took place in 2019 March at the University of Chicago.

Conversations:
% alphabetical by last name
Astro data group,


\appendix

\newpage
\bibliography{tess-world}


\end{document}
